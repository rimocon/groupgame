\documentclass{jarticle}
\usepackage[dvipdfmx]{graphicx}
\usepackage{listings,jlisting,url,here}

\lstset{%
  language={C},
  basicstyle={\small},%
  identifierstyle={\small},%
  commentstyle={\small\itshape},%
  keywordstyle={\small\bfseries},%
  ndkeywordstyle={\small},%
  stringstyle={\small\ttfamily},
  frame={tb},
  breaklines=true,
  columns=[l]{fullflexible},%
  numbers=left,%
  xrightmargin=0zw,%
  xleftmargin=3zw,%
  numberstyle={\scriptsize},% stepnumber=1,
  numbersep=1zw,%
  lineskip=-0.5ex%
}

% xなんちゃらが変数,進捗書く
\newcommand{\xe}{プレイヤーによる,敵への妨害行為の実装の検討}
% []内の数が引数,#数字で引数読む
\newcommand{\pitem}[1]{
\item #1
}

\title{個人レポート}
\author{6119019207 矢野大暉}
\date{2019/12/6 提出}

\begin{document}
\maketitle

\section{開発期間12/2~12/9の進捗状況} 
\subsection{進展事項}
今回の開発期間では,ゲームの設計として主に
\begin{itemize}
\pitem{\xe}
\end{itemize}
を行った.以下にその詳細を示す.

\subsection{\xe}
ゲームの最低限の機能は実装できたため,次に実装予定であるプレイヤーが「催涙スプレーをかける」という、敵に対して行う妨害行動の実装方法の検討をした.現在はプレイヤーの移動方向を取得できるように実装を行っている途中である.

\section{次回開発期間での予定}
\begin{itemize}
\item プレイヤーが敵に対して妨害をする処理の実装
\end{itemize}

次回は,前回の目標であった「敵の視界の実装」を実装したい.
\end{document}
