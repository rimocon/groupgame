\documentclass{jarticle}
\usepackage[dvipdfmx]{graphicx}
\usepackage{listings,jlisting,url,here}

\lstset{%
  language={C},
  basicstyle={\small},%
  identifierstyle={\small},%
  commentstyle={\small\itshape},%
  keywordstyle={\small\bfseries},%
  ndkeywordstyle={\small},%
  stringstyle={\small\ttfamily},
  frame={tb},
  breaklines=true,
  columns=[l]{fullflexible},%
  numbers=left,%
  xrightmargin=0zw,%
  xleftmargin=3zw,%
  numberstyle={\scriptsize},% stepnumber=1,
  numbersep=1zw,%
  lineskip=-0.5ex%
}


\title{個人レポート}
\author{6119019207 矢野大暉}
\date{2019/11/22 提出}

\begin{document}
\maketitle

\section{開発期間11/18~11/24の進捗状況} 
\subsection{進展事項}
今回の開発期間では,ゲームの設計として主に
\begin{itemize}
\item プレイヤーの移動の不具合を改善するための方法を検討
\item 敵の移動の実装
\item 参照バグの修正
\end{itemize}
を行った.以下にその詳細を示す.


\subsection{プレイヤーの移動の不具合を改善するための方法を検討}
プレイヤーに斜め移動をさせる際,上下左右の移動に比べて速度が早くなるため,その対策の検討をした.
実装方法としてプレイヤーの座標を実数で保存しておき,斜め移動の際に$\frac{1}{\sqrt2}$を掛けることによって斜め移動を実現できる可能性があることがわかった.

\subsection{敵の移動の実装}
敵の移動方向をマップの位置ごとに変える機能を実装した.
実装方法としては,マップ上に敵の動く方向を変える床を作成し,その床の真ん中に敵が乗ったときに,移動する方向を変更するように実装した.
敵の動くプログラムとしては,それぞれの敵が動く方向を持っていてその方向に動いているだけなので,移動する方向のみを変更することで良い.

\subsection{参照バグの修正}
プレイヤーの情報がカメラに入ってしまい,カメラの表示位置がおかしくなるというバグが発生した.
マップ作成機能において,必要なエラー処理が抜けていたためミスに気づくのに時間がかかったが,修正した.
マップ作成機能は私が実装したもので,メンバー全員が使うことになる機能であるので,もっとバグにすぐ気づけるようなエラー処理が必要であると感じた.


\section{次回開発期間での予定}
\begin{itemize}
\item 敵の視界の実装
\item 敵の移動パターンを増やす
\end{itemize}

今回の開発期間では,前回予定していた「敵の移動方向をマップごとに変更できる機能」の実装ができた.また,前回懸念していた敵の移動のズレも,それぞれのクライアントが同時に起動することで対策できることがわかった.

次回は,前回の目標であった「敵の視界の実装」を実装したい.
敵の移動パターンがマップに従うだけの単純なものとなっているので,次回それも強化したい.
また,今回自分がマップ作成機能にバグが発生したためそれを未然に防げるような例外処理をプログラムに付け加えたい.
\end{document}
