\documentclass{jarticle}
\usepackage[dvipdfmx]{graphicx}
\usepackage{listings,jlisting,url,here}

\lstset{%
  language={C},
  basicstyle={\small},%
  identifierstyle={\small},%
  commentstyle={\small\itshape},%
  keywordstyle={\small\bfseries},%
  ndkeywordstyle={\small},%
  stringstyle={\small\ttfamily},
  frame={tb},
  breaklines=true,
  columns=[l]{fullflexible},%
  numbers=left,%
  xrightmargin=0zw,%
  xleftmargin=3zw,%
  numberstyle={\scriptsize},% stepnumber=1,
  numbersep=1zw,%
  lineskip=-0.5ex%
}
% xなんちゃらが変数,進捗書く
\newcommand{\xa}{監視カメラのハッキングのゲージ実装}
\newcommand{\xb}{敵の視界実装}
\newcommand{\xc}{催涙スプレーの実装}
\newcommand{\xd}{一番近いプレイヤーに追尾する敵の動きの実装}
% []内の数が引数,#数字で引数読む
\newcommand{\pitem}[4]{
\item #1
\item #2
\item #3
\item #4
}

\title{グループレポート}
\author{6119019092 織田武瑠 6119019207 矢野大暉 6119019056 山口力也}
\date{\number\year/\number\month/\number\day 提出}

\begin{document}
\maketitle

\section{開発期間12/16~12/22の進捗状況} 

\subsection{進展事項}
今回の開発期間では,ゲームの設計として主に
\begin{itemize}
\pitem{\xa}{\xb}{\xc}{\xd}
\end{itemize}
を行った.以下にその詳細を示す.

\subsection{\xa}
ハッキング動作を行うときの長押ししている時間を表すゲージを実装した.長押ししている間にゲージが溜まっていき、指定した時間以上長押しするとゲージが最大になる.ゲージが溜まりきっているときにボタンを離すと、ハッキング動作を行う.

\subsection{\xb}
敵の視界を実装した.本ゲーム内では、この視界内に金塊を持ったプレイヤーが入ると、ゲームオーバーになる.

\subsection{\xc}
催涙スプレーを実装した.敵にこの催涙スプレーが当たると、敵の動きが一定時間止まり、視界がなくなる.一定時間を過ぎると復活する.

\subsection{\xd}
敵の動きのパターンに新しく、敵から最も近いプレイヤーを追尾する動きを実装した.
プレイヤーと敵の間に壁がある場合は迂回し、一定時間同じ場所にとどまると、ランダムな動きをする.

\end{document}
