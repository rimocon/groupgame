\documentclass{jarticle}
\usepackage[dvipdfmx]{graphicx}
\usepackage{listings,jlisting,url,here}

\lstset{%
  language={C},
  basicstyle={\small},%
  identifierstyle={\small},%
  commentstyle={\small\itshape},%
  keywordstyle={\small\bfseries},%
  ndkeywordstyle={\small},%
  stringstyle={\small\ttfamily},
  frame={tb},
  breaklines=true,
  columns=[l]{fullflexible},%
  numbers=left,%
  xrightmargin=0zw,%
  xleftmargin=3zw,%
  numberstyle={\scriptsize},% stepnumber=1,
  numbersep=1zw,%
  lineskip=-0.5ex%
}
% xなんちゃらが変数,進捗書く
\newcommand{\xa}{催涙スプレーのゲージを表示}
\newcommand{\xb}{プレイヤーの画像が上下左右と斜めで変わるようになった}
\newcommand{\xc}{ハッキング動作のバグの修正}
% []内の数が引数,#数字で引数読む
\newcommand{\pitem}[3]{
\item #1
\item #2
\item #3
}

\title{グループレポート}
\author{6119019092 織田武瑠 6119019207 矢野大暉 6119019056 山口力也}
\date{\number\year/\number\month/\number\day 提出}

\begin{document}
\maketitle

\section{開発期間12/23~1/19の進捗状況} 

\subsection{進展事項}
今回の開発期間では,ゲームの設計として主に
\begin{itemize}
\pitem{\xa}{\xb}{\xc}
\end{itemize}
を行った.以下にその詳細を示す.

\subsection{\xa}
催涙スプレーに使用時間の制限をかけて,残りのスプレー残量をゲージで表示した.

\subsection{\xb}
プレイヤーがゲームパッドの入力を受けて上下左右,斜めへの移動時に,プレイヤーの画像が変わるようになった.

\subsection{\xc}
カメラのハッキング動作の際のボタン長押し中にバグが発生していたため,修正した.

\end{document}
