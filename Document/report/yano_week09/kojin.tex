\documentclass{jarticle}
\usepackage[dvipdfmx]{graphicx}
\usepackage{listings,jlisting,url,here}

\lstset{%
  language={C},
  basicstyle={\small},%
  identifierstyle={\small},%
  commentstyle={\small\itshape},%
  keywordstyle={\small\bfseries},%
  ndkeywordstyle={\small},%
  stringstyle={\small\ttfamily},
  frame={tb},
  breaklines=true,
  columns=[l]{fullflexible},%
  numbers=left,%
  xrightmargin=0zw,%
  xleftmargin=3zw,%
  numberstyle={\scriptsize},% stepnumber=1,
  numbersep=1zw,%
  lineskip=-0.5ex%
}
% xなんちゃらが変数,進捗書く
\newcommand{\xb}{催涙スプレーのゲージを表示}

% []内の数が引数,#数字で引数読む
\newcommand{\pitem}[1]{
\item #1
}

\title{個人レポート}
\author{6119019207 矢野大暉}
\date{\number\year/\number\month/\number\day 提出}

\begin{document}
\maketitle

\section{開発期間12/23~1/19の進捗状況} 

\subsection{進展事項}
今回の開発期間では,ゲームの設計として主に
\begin{itemize}
\pitem{\xb}
\end{itemize}
を行った.以下にその詳細を示す.

\subsection{\xb}
催涙スプレーに使用時間の制限をかけて,残りのスプレー残量をゲージで表示した.
使用時間は定数で指定することで自分以外の同じグループのメンバーでもかんたんに時間を指定できる.

\section{次回開発期間での予定}
次回は全員のプログラムを統合して,不具合がないかを確認したい.
\end{document}
